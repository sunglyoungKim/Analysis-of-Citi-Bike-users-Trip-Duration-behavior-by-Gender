IDEA:
Women has short trip duration of usage of citibike than men.

NULL HYPOTHESIS:

The women's trip duration that use of citi-bike is the same or higher than the men's trip duration of usage of the bike.

\section{$H_0$ : $\frac{F_{Wd}}{W_{n}} >= \frac{F_{Md}}{M_{n}} $ }



\section{$H_1$ : $\frac{F_{Wd}}{W_{n}} < \frac{F_{Md}}{M_{n}} $}

where $F_{Wd}$ is Frequency of Women trip duration, $F_{Md}$ is Frequency of Men trip duration, ${W_{n}}$ is total number of Women, and ${M_{m}}$ is total number of Men

or identically:

$H_0$ : $\frac{F_{Wd}}{W_{n}} - \frac{F_{Md}}{M_{n}} $ >= 0

$H_1$ : $\frac{F_{Wd}}{W_{n}} - \frac{F_{Md}}{M_{n}} $ < 0

\subsection{The significance level is chosen to $\alpha=0.05$}

\subsection{which means i want the probability of getting a result at least as significant as mine to be less then 5 \%}


The citi-bike is a bike sharing system that is run by Motivate in NYC, according to Citi-Bike homepage\cite{nyc}. The system has been operating since May 2013 and it has been operated for 24 hours/day, 7 days/week, 365 days/year because the company installed 600 bike station over 55 neighborhood in NYC \cite{nyc}. Since the sharing system is embeded on NYC the new type of transportation opens new way of one-way trips, such as commute 

It's fun, efficient and affordable – not to mention healthy and good for the environment.
Citi Bike, like other bike share systems, consists of a fleet of specially designed, sturdy and durable bikes that are locked into a network of docking stations throughout the city. The bikes can be unlocked from one station and returned to any other station in the system, making them ideal for one-way trips. People use bike share to commute to work or school, run errands, get to appointments or social engagements, and more.
Citi Bike is available for use 24 hours/day, 7 days/week, 365 days/year, and riders have access to thousands of bikes at hundreds of stations across Manhattan, Brooklyn, Queens and Jersey City." 